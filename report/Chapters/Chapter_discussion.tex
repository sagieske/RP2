% Chapter X
\chapter{Conclusion} % Chapter title
\label{ch:cncl} % For referencing the chapter elsewhere, use \autoref{ch:name} 
%This chapter describes the conclusions that can be drawn from this research. In \autoref{sec:con} the results gained in this research will be discussed and in \autoref{sec:fut} future work for improvement of this research are described.

%\section{Conclusion}

In this research, a decision tree learning approach is used for camera identification with the use of JPEG quantization tables. 

Images are retrieved in forensic investigation as important evidence. In these cases, the origin of these images need to be identified with the use of camera identification models. The set of JPEG quantization tables used for JPEG compression in the camera can be used for these models. Its matching process for tables with an existing database can be time costly when large sets of images are recovered for forensic investigation. For this reason, this research has focused on reducing the search space for JPEG quantization tables. As the camera identification problem can be seen as a pattern recognition problem, machine learning techniques can be applied. The research question on which this research has focused was stated as: \textit{`Can searching through JPEG quantization tables be optimized with the use of decision tree learning?'}

%With a growing popularity of taking pictures, the amount of images found in forensic investigations is large. In certain cases, these images can be important evidence and the origin of the images need to be identified. Camera identification models focus on identification of the camera source of an image and make use of several hardware features of the camera. One of these features is the set of JPEG quantization tables used for JPEG compression in the camera. It is reasonably effective at narrowing the source of an image. Each set of tables consists of 128 values and the matching process for tables with an existing database can be time costly when large sets of images are recovered for forensic investigation. This research has focused on reducing the search space for JPEG quantization tables.

In this research, a decision tree learning approach is taken. Supervised machine learning is used to create a decision tree model to predict camera makes and models on basis of the set of JPEG quantization tables that belong to an image. First, a feature selection procedure is performed in which the identifiable parameters in a set of JPEG quantization tables are be reduced from 128 to 50 attributes. These 50 parameters do not show a clear special correlation with the tables. 

Then, the decision tree model is trained and evaluated using the weighted F$_2$-score in a 5-fold stratified cross-validation. The decision tree model is compared to two different prediction models that use a hash database. The decision tree model gains a the highest F$_2$-score (89\%) for the prediction of the camera make. The 1$\rightarrow N$ hash database model performs slightly better for the prediction of the camera make than the decision tree model, F$_2$-score of 83\% and F$_2$-score of 80\% respectively. However, it gains low precision rates because it returns all possible classes. The low precision indicates that the search space is not effectively decreased. In contrast, the decision tree model gains high rates for accuracy while maintaining a high precision. The decision tree model can also handle small differentiations in tables better than the hash database models.

Overall, the decision tree learning algorithm gains a good performance for the prediction of the camera make as well as for the prediction of the camera model. 

\section{Future Work}\label{sec:fut}
The following adjustments for the methods used in this research, as well as improve-
ments using other techniques, are proposed:
\begin{itemize}
\item \textbf{Extend image database}: This research can be extended by using a larger image database which comprises of more different camera make/models.
\item \textbf{Extend feature set}: The feature set can be extended with more attributes that correspond to the image. These can be related to the JPEG quantization tables, but also to other meta-data retrieved from the image.
\item \textbf{Compare to other learning algorithms}: The decision tree learning algorithm is prone to overfitting. Other supervised machine learning algorithms, such as Naive Bayes and Support Vector Machine, tend to create better generalizations and are less susceptible for overfitting. Future research can be performed on the performance of other machine learning algorithms in comparison with the decision tree learning algorithm.
\item \textbf{Probabilistic classification}: Because a set of JPEG quantization tables can correspond to multiple camera make/models, this research can be improved with the implementation of probabilistic classification. This method gives a probability distribution over a set of classes instead of predicting one single class for a feature set.

\end{itemize}
%----------------------------------------------------------------------------------------
\iffalse
\section{Conclusion}\label{sec:con}
This research has focused on predicting popular activities using Twitter data. This task was divided into 3 subtasks: activity classification, activity extraction and ranking. The subtask of activity classification is examined with the use of Naive Bayes and Support Vector Machine (SVM) classification, experimenting with several values for feature vectors. The best classifier scored 75 \% on the F1-scoring, which was a SVM using a RBF kernel and, among other things, standard tokens and a combination of uni- and bigrams.
The subtask of activity extraction is examined using two different methods; the Latent Dirichlet Allocation (LDA) algorithm and the POS tag approach. The LDA algorithm performs poorly for this task as it clusters words which are not interlinked for the same activity. The algorithm is dependent on meaningful words to perform clustering, however, the words created using the log-likelihood distribution are not significantly meaningful. The data set used for this task is sparse and results in unsatisfactorily performance.
However, the POS tag approach does create satisfactorily results with a scoring of 66 \% for extraction of activities. It identifies activities using frequent POS tag sequences from a training set. 
The ranking for the popularity of activities is conducted using a frequency count for activities. In the top 40 for most popular activities for one evening 75 \% showed a correct activities. The ranking demonstrates common, as well as uncommon popular activities. Activity descriptions are, however, not clustered into groups concerning the same activity.
Overall, the conclusion can be drawn that popular activities using Twitter data can be predicted satisfactorily accurate, although there is room for improvements. 
\section{Future Work}\label{sec:fut}
The following adjustments for the methods used in this research, as well as improvements using other techniques, are proposed:
\begin{itemize}
\item \textbf{More training data classification}: In the classification of activities, false predictions are often a result of the fact that activities can be described in many different ways. In order to classify these predictions, the classifier has to have more training data in order to encounter and learn these descriptions.
\item \textbf{More annotators}: As seen in the results, the agreed assesement of the annotation is initially very low and increases after incorporating rules. In order to identify more of these rules and improve the annotations, more annotators can be used.
\item \textbf{Clustering activity words}: In order to improve the extraction of activities described by similar words, a POS tag analysis can be conducted to assign words and their conversions as, for example, dimmunitives, to the same clusters to improve the prediction of the popularity of these activities.
\item \textbf{Different time frame}: The time frame on which this research focused was `vanavond'. The system can be extended by examining different time frames.
\item \textbf{Combining different sources}: This research relied soley on data obtained from Twitter. As an improvement for the system different sources can be combined, such as other social media platforms for the identification of more popular activities and more accurate calculations in ranking, or informational websites, to improve the identification of activities in Twitter messages.
\end{itemize}
\fi
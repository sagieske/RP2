% Chapter X

\chapter{Introduction} % Chapter title

\label{ch:name} % For referencing the chapter elsewhere, use \autoref{ch:name} 


%----------------------------------------------------------------------------------------

\section{Motivation}

Taking pictures is very easy and popular in this digital age. The demand for digital cameras was forecast to be 86 million units for 2013 by Futuresource Consulting\cite{futuresource}. And even though a decline in market share is present for digital cameras, due to the  proliferation of smartphones (nowadays all equipped with a camera function) the total number of digital images taken each year is very high. Social media sites which have photo upload functions, such as Facebook and Instagram, report significantly huge numbers on the total upload of images. Facebook alone reported in a white paper \cite{whitefacebook} that more than 250 billion photos are uploaded to their site, with on average a total upload of more than 350 million photos every day. Statistics on Instagram\footnote{http://instagram.com/press/ accessed 03-06-2014} show a total of 20 billion photos shared on Instagram.

Due to this popularity, digital images are often recovered in forensic investigation. For example, in child pornography cases many digital images are present and are important evidence for the investigation. In such a case it can be very important to identify the origin of images to a specific camera or identify images that come from a common source. This can be done by uncovering traces on pictures that are distinguishable for camera models. One of these traces is the JPEG quantization table, which is specified as a set of 8 $\times$ 8 (integer) values. Separate quantization tables are employed for luminance and chrominance data, where some implementations include two chrominance quantization tables for chrominance-red and chrominance-blue. 

In order to match JPEG quantization tables a comparison between 128 values, or 192 when two chrominance quantization tables are present, is made. With over a dozen different camera brands, each developing different models over the years, the number of camera models (and consequently the number of JPEG quantization tables) to be matched against is significantly high. The matching of large databases of images against these camera models will be time costly as for every matching 128 or more integer comparisons are made. This matching process needs to be minimized since time is often limited in forensic investigations. This research will focus on optimizing search through the image databases regarding JPEG quantization tables.
%-----------------------------------------------

\section{Focus of research}
The research question on which is focused is set as: \textit{`How can searching through JPEG quantization tables be optimized?}

In order to answer the research question, this research will focus on the following subquestions:
\begin{enumerate}
\item What are identifiable parameters of JPEG quantization tables?
\item How can we quickly decrease the search space for JPEG quantization table matches?
\end{enumerate}
In the following chapters the answers to these questions will be given.
\iffalse
\section{Challenges}\label{challenges}

There are different challenges in the task of activity prediction using Twitter data. Below, we will list the challenges:

\begin{description}
\item[Short texts] Twitter messages contain only 140 characters or less which results in sparse data for text classification.
\item[Informal style] Twitter is used in an informal matter to express oneself in few words. Twitter messages have an informal structure; words are abbreviated or omitted creating messages which are often not grammatically correct. Due to this informal style, Natural Language Processing (NLP) tools perform extremely poor.
\item[Changing vocabulary] As a result of the informal style Twitter messages contain a significant amount of slang. A constantly changing vocabulary is used for the creation of these messages. A dynamic program has to be designed to accommodate these changes.
\item[Activity extraction] Even though Twitter messages have a small size, users often try to comment as much as possible in these messages which results in short messages containing multiple activities. Also, due to the informal language, important verbs which indicate an activity are ommitted.
\item[Time-awareness] Twitter messages can contain activities which happen in different time frames. Due to the short text size and the informal style only few temporal words are used. This results in difficulties in identifying time frames of activities.
\end{description}

Based on the challenges mentioned above, an activity prediction system should be dynamic to handle the constantly changing vocabulary and informal language. The system should overcome classification challenges due to the limited size of texts. Left out verbs should be taken into account for activity extraction. Finally, the system should take into account the aspect of time-awareness by only predicting popular activities using extraction for activities taking place in the future.
\fi
%------------------------------------------------



%----------------------------------------------------------------------------------------

% Chapter X

\chapter{Results} % Chapter title

\label{ch:results} % For referencing the chapter elsewhere, use \autoref{ch:name}

In this chapter the results of this research will be discussed. %\autoref{sec:res_activity} shows the results of the activity classification. In \autoref{sec:res_extraction} the results of the two different activity extraction methods are described and finally, in \autoref{sec:res_ranking}, the results for ranking popular activities are shown.

\section{Extraction of JPEG quantization tables}
In total there are 1,016 unique sets of JPEG quantization tables retrieved from the images. In \autoref{tab:dqt_count} the number of unique JPEG quantization tables per camera model are shown. As can be derived, there are JPEG quantization tables that are found for more than one camera model. There are 398 quantization tables found in images with different camera models. 

As the chrominance color space can be divided into chrominance-red and chrominance-blue, it could have occurred that 3 JPEG quantization tables are retrieved for an image. However, only sets of 2 JPEG quantization tables are found in the dataset.

\section{Feature Selection}
The two JPEG quantization tables are converted to a set of features and the extra statistical features (as described in \autoref{sec:featselect}) are added. Each JPEG quantization table contains 64 values and 105 extra statistical features per table are added, which gives a total of 338 attributes per image. The extra features did not have impact on the scores for the evaluation. Therefore, these extra statistical features are omitted from the feature set.

After the feature selection procedure the identifiable parameters for the decision tree learning algorithm are reduced to 50 parameters. 

\section{Decision Tree Learning}
The decision tree learning algorithm has created a decision tree of 603 nodes with a depth of 26 nodes. The average F$_\beta$-score for prediction of the camera make is 89\% and for prediction of the camera model 80\%. In the following subsections the results of the decision tree evaluation is discussed as well as the comparison against the hash database models.

\subsection{Evaluation}

\subsection{Comparison against hash database}
\begin{table}
\begin{tabular}{| c| c| c| c|}
\hline
Algorithm & Precision & Recall & F2-score\\
\hline
Hash (1-1) & 79 \% & 68 \% & 68 \%\\
Hash (1-n) & 50 \% & 99 \% & 83 \%\\
Decision tree & 90 \% & 89 \% & 89 \% \\
\hline
\end{tabular}
\caption{Camera Make Identification}
\end{table}

%1300 tables which have mutliple possibilities and thereby increasing the search space for fuurther research
\begin{table}

\begin{tabular}{| c| c| c| c|}
\hline
Algorithm & Precision & Recall & F2-score\\
\hline
Hash (1-1) & 54 \% & 39 \% & 37 \%\\
Hash (1-n) & 50 \% & 98 \% & 83 \%\\
Decision tree & 78 \% & 82 \% & 80 \% \\
\hline
\end{tabular}
\caption{Camera Model Identification}

\end{table}

\section{Discussion}

Since a decision tree uses a one-to-one mapping of a JPEG quantization table to a camera make/model it will not perform perfectly on tables that occur at multiple camera make/models. The classifier makes a choice to which camera make/model this table is mapped. 

%----------------------------------------------------------------------------------------


% Chapter X

\chapter{Method} % Chapter title

\label{ch:method} % For referencing the chapter elsewhere, use \autoref{ch:name} 


%----------------------------------------------------------------------------------------

In order to optimize search through JPEG quantization tables the search space needs to be decreased. This reduction in search space can be performed by creating a decision tree model. This model maps observations about an item (specific features of the quantization table) to conclusions about the item's target value (camera model). Decision tree learning is used, which is the construction of a decision tree from class-labelled training tuples, to identify important parameters and their position in the decision tree model. The matching with the use of decision tree model parameters and the matching between full JPEG quantization tables are both benchmarked for time to see whether the search time is accelerated.

The following steps are taken:
\begin{enumerate}
\item Gather dataset of JPEG quantization tables. Dataset of pictures and their JPEG quantization table and for JPEG quantization for camera models are needed.
\item Create numerous possible parameters to identify these tables. Rewrite JPEG quantization table as collection of these parameter values.
\item Create training and test set for decision tree learning.
\item Perform decision tree learning to create decision tree model
\item Perform benchmarks: matching with the decision tree model parameters and matching full JPEG quantization tables
\end{enumerate}

\section{JPEG Quantization Table}

EXPLAIN BASICS
% Chapter X

\chapter{Related Work} % Chapter title

\label{ch:literature} % For referencing the chapter elsewhere, use \autoref{ch:name} 


%----------------------------------------------------------------------------------------

Research on digital image forensics is a growing field. It focuses on two main interests, namely source identification and forgery detection. Van Lanh et al. \cite{van2007survey} created a survey on digital camera forensics, which describes several techniques in these two fields. Their survey shows the use of intrinsic features of camera hardware and software for camera identification and concludes that hardware features give more reliable and better result. To distinguish between cameras of the same model imperfections of camera the use of hardware features seems to be the best method. Methods for forgery detection also rely on hardware-dependent characteristics but show lower accuracy rates compared to camera identification methods. In another survey, Weiqi et al. \cite{luo2007survey} describe methods for passive technology for digital image forensics. They state that in most cases passive forensics can be converted to a problem of pattern recognition.

In forgery detection methods to identify JPEG quantization tables are often used. In research by Kornblum\cite{kornblum2008using} quantization tables used by several image software are identified. A software library called Calvin is developed to identify those images who cannot
be guaranteed to have been created by a real camera. Research by Farid\cite{4773149} shows a technique for detecting tampering in low-quality JPEG images by identifying a cumulative effect of quantization.

JPEG quantization tables can also be used for source identification. Farid has performed research\cite{farid1}\cite{farid2008digital} on source identification with the use of JPEG quantization tables. This research states that a sort of camera signature is embedded within each JPEG image due to the used JPEG quantization tables since they differ between manufacturers. Although the JPEG quantization is not perfectly unique, the majority of cases where the same tables are found it is cameras from the same manufacturer that share the same quantization table. It states that (the use of JPEG quantization tables) ``\textit{is reasonably effective at narrowing the source of an image to a single camera make and model or to a small set of possible cameras.}" (p. 3)

There exist several projects where JPEG quantization tables are used as camera signatures. For example, the JPEGsnoop\footnote{http://www.impulseadventure.com/photo/jpeg-snoop.html} project reports a huge amount of information to expose hidden information in images. Another project is the (discontinued) commercial FourMatch\footnote{http://www.fourandsix.com/fourmatch}, which was focused on forgery detection. These projects compare the camera signature found for an image with a database of camera signatures to identify the camera make and model. These projects are not focused on matching large sets of images against a large camera database. In contrast, this research hopes to contribute by creating a decision tree model in order to decrease the search space for large datasets and which can easily be combined further with other (more accurate) source identification techniques. 
% Chapter X

\chapter{Related Work} % Chapter title

\label{ch:literature} % For referencing the chapter elsewhere, use \autoref{ch:name} 


%----------------------------------------------------------------------------------------

Information extraction has recently become the subject of several studies involving social media. 
This is the task of automatically extracting structured information from unstructured machine-readable documents. An example of an information extraction research focused on social media is the research of Hua et al.\cite{hua_overview}, who conducted a survey on information extraction for microblogs in three types of information concepts, presenting an overview of approaches used in research for extracting personal, social and travel information. Another example is the research of Vieweg et al.\cite{vieweg}, who examined the contribution to situational awareness of microblogs employing information extraction. Information extraction is also used widely in the research for a variety of prediction tasks in different domains, to which the task of this research can be assigned.

\section{Prediction Tasks}
Prediction tasks are subject to many researches in information extraction. In the research of Jiang et al. \cite{jiang} topic-sentiment is predicted using Twitter data. Classification tasks are conducted by implementing support vector machines with linear kernels and extending classification with target-dependent sentiment features. The results show target-dependent sentiment classification outperforms target-independent classifiers. 

Another example is the research of Tumasjan et al. \cite{Tumasjan} in which the prediction of election outcomes from Twitter data are conducted, thus demonstrating that this data can be considered a valid indicator of political opinion. This prediction task is approached by analysis of sentiment conducted by LIWC2007, a text analysis software developed to assess emotional, cognitive, and structural components of text samples using a psycho-metrically validated internal dictionary. 

Research on stock market predictions using Twitter data is another example for research focused on prediction task and was conducted by Bollen et al. \cite{Bollen}, who implemented two algorithms on mood indication, namely, OpinionFinder and Google-Profile of Mood States (GPOMS). Results show accurate predictions of closing values of Dow Jones Industrial Average using sentiment analysis,

Another popular domain is the prediction of movie revenues, in which initial work is done by among other Asur et al. \cite{asur} and Joshi et al. \cite{joshi} using lineair regression. These studies show strong correlation between the amount of attention a topic has attracted on Twitter and its ranking in the opening weekend revenue. In this domain, another research is conducted by Oghina et al. \cite{Oghina} for the prediction of IMDB movie ratings using cross-channel social media signals. The research identifies qualitative and quantitative indicators for a movie in social media and employs and the results show that the best performing model is able to create ratings very closely to observed data.

Go et al.\cite{go} conducted research on automatic classification of sentiment of Twitter data exploring n-gram features. The machine algorithms Naive Bayes, Maximum Entropy and SVM were applied, achieving classification results with  a level of accuracy of above 80\%. Another study on sentiment analysis of Twitter data was conducted by Pandey and Iyer \cite{pandey}, which examined classification based on machine learning techniques. Different feature selections and classification algorithms were implemented, achieving highly accurate levels for sentiment analysis employing SVM and Naive Bayes algorithms for classification.

\section{Temporal Predictions}
Recently, research in prediction tasks has been focused on temporal predictions. For example, Ritter et al.\cite{Ritter} implemented an open-domain event extraction method, extracting and categorizing significant events. Their research adopted an approach based on latent variable models combining modeling selectional preferences and unsupervised information extraction. Ritter focused on large events due to take place several days in the future. The research of H\"urriyeto\v{g}lu, et al. \cite{hurriye} focuses on estimating the remaining time between a series of micro texts (Twitter messages) and the future event referred to via a hashtag. This research concerned potentially large-scale news events and employed regression algorithms using local and linear regression. The results gave a fairly accurate prediction for events due to take place within four days and shows a promising prediction system.

\section{Activity Predictions}
Only very recently, research is done on activity prediction and therefore there are only been a few studies regarding this task. An example is the research by Song et al. \cite{activity_colab} which proposes a collaborative boosting learning framework using local classifiers for microblogs. The collaboration of two classification methods is used for the recognition of user activity. In this research the focus is put on modeling of activities towards user profiles, exploiting information gained on the connections between users for accurate activity classification. Results show the algorithm outperforms several state-of-the-art algorithms and baselines. Even though this study focuses on activity recognition, it does not focus on a temporal prediction and is mainly focused on activities of an individual user, in contrast to the focus of this research, which is put on trying to identify future activities and focuses on the overall popularity of future activities. Another contrast between the two research is that the dataset of Song et al.contains Chinese messages, whereas this research uses a dataset with Dutch messages.

The research which is closest to the research conducted for this paper is that of Weerkamp and De Rijke \cite{Weerkamp}, who investigated the possibility of activity prediction. This research conducted an initial exploration of the activity prediction task for trying to predict future plans of people, by establishing a set of activities that are likely to become popular at a later time using Twitter data. In this exploration an activity word that indicates a future activity was used for the selection of Twitter messages. That such prediction is possible was affirmed by their results. Our research builds on on the research of Weerkamp and De Rijke and will be used as base for further research.

% Chapter X
\chapter{Experimental Setup} % Chapter title
\label{ch:impl} % For referencing the chapter elsewhere, use \autoref{ch:name} 
%----------------------------------------------------------------------------------------
In this chapter an overview is given of the approach to use decision tree learning for optimizing search in JPEG quantization models. It describes the several steps and their implementations in this research. 

\section{Dataset}
The dataset which is used in this research consists of 45,666 images. These images are retrieved from the Dresden Image Database \cite{Gloe:2010aa} and an image database from the Netherlands Forensic Institute. This dataset has images from 19 different camera makes and a total of 41 different camera models. The camera make and models are listed in \autoref{tab:makemodel} together with the number of pictures made with these cameras. Next to regular digital cameras there are also images from other types of cameras included in the dataset, such as images taken by smartphones (e.g. Blackberry), a webcam (e.g. Logitech), scanners (Epson) and a Playstation device (PS Vita). 

\section{Approach}

%In order to optimize search through JPEG quantization tables the search space needs to be decreased. This reduction in search space can be performed by creating a decision tree model. This model maps observations about an item (specific features of the quantization table) to conclusions about the item's target value (camera model). Decision tree learning is used to identify important parameters and their position in the decision tree model. The matching with the use of decision tree model parameters and the matching between full JPEG quantization tables are both benchmarked for time to see whether the search time is accelerated.
In this section the approach is given for the prediction of the camera make as well as the camera model based on the JPEG quantization tables.
First, all JPEG quantization tables are extracted from the images and stored with the corresponding camera make or model label. Next, these tables are converted to simple feature sets and several extra features are added. On these feature sets a feature selection is performed to retrieve the most important features. The set of important features and corresponding labels is split into a training set and a test set. The training set is used as input for in the decision tree classifier which returns a decision tree model. This model is then used on the test set in order to evaluate its performance. In addition, two prediction models, using a database in which JPEG quantization tables are hashed and stored with their labels, are created. In order to give a good view on the performance of the decision tree classifier, its performance is compared with the performance of these two prediction models.
\\~\\
The following steps are taken:
\begin{enumerate}
\item Extract JPEG quantization tables from images
\item Generate feature set for JPEG quantization tables
%\item Create numerous possible parameters to identify these tables. Rewrite JPEG quantization table as collection of these parameter values.
\item Train decision tree classifier 
\item Evaluate classifications
\item Compare against method using hash database
\end{enumerate}


\subsection{Extraction of JPEG quantization tables}
As described in \autoref{sec:dqt} the JPEG quantization tables are used during JPEG compression and relate to the compression ratio of an image. These tables are saved in JFIF headers and can be extracted from the JPEG file. In this research the \textit{djpeg} \footnote{http://linux.about.com/library/cmd/blcmdl1\_djpeg.htm} tool is used. This tool receives an image as input and can output the JPEG quantization tables. These tables are then retrieved with the use of a python script. The camera make and models are retrieved from the filenames and are stored together with their JPEG quantization tables for further processing.

\subsection{Feature selection}

\subsection{Decision tree learning}

\subsection{Evaluation}\label{sec:eval}
The performance of the prediction models is evaluated with the use of the the F$_\beta$-score. This score is a measure for the accuracy of a test and considers both precision and recall. The $\beta$ parameter can be set to let the user give more weight to recall ($\beta > 1 $) or precision ($\beta < 1 $). The formula is described in \autoref{eq:1}.

\begin{equation}\label{eq:1}
precision =  \frac{ \left\vert{\left\{ \text{ relevant documents} \right\} \cap \left\{ \text{ retrieved documents} \right\}}\right\vert }{ \left\vert{\left\{ \text{ retrieved documents} \right\}}\right\vert}
\end{equation}

\begin{equation}\label{eq:1}
recall = \frac{ \left\vert{\left\{ \text{ relevant documents} \right\} \cap \left\{ \text{ retrieved documents} \right\}}\right\vert }{ \left\vert{\left\{ \text{ relevant documents} \right\}}\right\vert}
\end{equation}


\begin{equation}\label{eq:1}
F_\beta = 1 + \beta^{2} * \frac{precision * recall}{(\beta^{2} *precision) + recall}
\end{equation}

In this research both the precision and recall are important: precision is important to generate a smaller search space, this measure concerns the fraction of retrieved images that are actually correct; recall is important to retrieve all possible incriminating images, this measure concerns the fraction of relevant images that are actually retrieved. With regard to forensic investigations the recall of images is very important because you want to gather as much incriminating images as possible. For this reason $\beta$ is set to 2 to give a higher weight to recall. 

\subsection{Comparison against hash database}
A simple way of predicting camera make and model according to JPEG quantization tables is to build a database which contains encountered JPEG quantization tables and their corresponding make and model, and then query for the found JPEG quantization tables. Since these tables are comprised of many variables, it is more efficient to store them as a single hash signature. The JPEGSnoop software, for example, works with a database of signatures. In order to evaluate the decision tree learning algorithm, its performance is compared to the performance when a database containing hashes is used. The hash database method is trained and evaluated with the same subsets that are used for the decision tree learning algorithm. 

The hash database is created by hashing every set of JPEG quantization tables with the SHA256 hashing algorithm and then saving this in the database with its corresponding label. Two different implementations are made:
\begin{enumerate}
\item \textbf{1$\rightarrow$1 Hash Database}: a 1$\rightarrow$1 mapping of a set of JPEG quantization tables to 1 camera make/model. The JPEG quantization table is mapped to the first camera make/model that is encountered.
\item \textbf{1$\rightarrow N$ Hash Database}: a 1$\rightarrow N$ mapping of a set of JPEG quantization tables to multiple possible camera make/model. The JPEG quantization table can belong to different camera make/models. This method is also used in the JEGSnoop software. 
\end{enumerate}

Both hash database methods are evaluated with the F$_\beta$-score as described in \autoref{sec:eval}.
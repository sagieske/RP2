
\title{{\normalsize Project Proposal RP2}\\Search optimization for JPEG quantization tables\\ }
\author{
		Sharon Gieske (6167667)  \\
}
\date{\today}

\documentclass[a4paper,8pt]{article}
\usepackage{cite}
\usepackage{titling}
\usepackage{graphicx}
\usepackage[hyperfootnotes=true]{hyperref}
\usepackage{float} % for table
\restylefloat{table}
\usepackage{enumitem}
\setlist{itemsep=0.1pt,leftmargin=*}
\usepackage[margin=1.5in]{geometry}
\usepackage{nameref}

\setlength{\droptitle}{-1.5in}

\begin{document}
\maketitle
\vspace{-2em}
\section*{Introduction}
Taking pictures is very easy and popular in this digital age. The demand for digital cameras was forecast to be 86 million units for 2013 by Futuresource Consulting\cite{futuresource}. And even though a decline in market share is present for digital cameras, due to the  proliferation of smartphones (nowadays all equipped with a camera function) the total number of digital images taken each year is very high. Social media sites which have photo upload functions, such as Facebook and Instagram, report significantly huge numbers on the total upload of images. Facebook alone reported in a white paper \cite{whitefacebook} that more than 250 billion photos are uploaded to their site, with on average a total upload of more than 350 million photos every day. Statistics on Instagram\footnote{http://instagram.com/press/ accessed 03-06-2014} show a total of 20 billion photos shared on Instagram.

Due to this popularity, digital images are often recovered in forensic investigation. For example, in child pornography cases many digital images are present and are important evidence for the investigation. In such a case it can be very important to identify the origin of images to a specific camera or identify images that come from a common source. This can be done by uncovering traces on pictures that are distinguishable for camera models. One of these traces is the JPEG quantization table, which is specified as a set of 8 $\times$ 8 (integer) values. Separate quantization tables are employed for luminance and chrominance data, where some implementations include two chrominance quantization tables for chrominance-red and chrominance-blue. 

In order to match JPEG quantization tables a comparison between 128 values, or 192 when two chrominance quantization tables are present, is made. With over a dozen different camera brands, each developing different models over the years, the number of camera models (and consequently the number of JPEG quantization tables) to be matched against is significantly high. The matching of large databases of images against these camera models will be time costly as for every matching 128 or more integer comparisons are made. This matching process needs to be minimized since time is often limited in forensic investigations. This research will focus on optimizing search through the image databases regarding JPEG quantization tables.

% digital image forensic
% matching images to camera model. clustering images (common source of images)
% QT effective at narrowing the source ofan image to a single camera make and model or to a small set of possible cameras



%Acceleration methods for searching image databases, for example through optimizing search through quantization tables in JPEG. Some investigation has been done on how this JPEG characteristic can be used by such methods, but further investigation should give a better view on its feasibility. Other JPEG characteristics not yet exploited by any search method in current use may be investigated as well. These methods are used to search for images that have, for example, deviant or specific values for these characteristics. Certain values may indicate the use of a camera of some kind, or that it has been altered (or recreated) by specific image editing software. A proof-of-concept that shows the use of such characteristics in search methods will probably be implemented.


\section*{Research Question}

The research question on which is focused is set as: \textit{`How can searching through JPEG quantization tables be optimized?}

In order to answer the research question, this research will focus on the following subquestions:
\begin{enumerate}
\item What are identifiable parameters of JPEG quantization tables?
\item How can we quickly decrease the search space for JPEG quantization table matches?
\end{enumerate}

\section*{Related Work}
Research on digital image forensics is a research growing field. It focuses on two main interests, namely source identification and forgery detection. Van Lanh et al. \cite{van2007survey} created a survey on digital camera forensics, which describes several techniques in these two fields. Their survey shows the use of intrinsic features of camera hardware and software for camera identification and concludes that hardware features give more reliable and better result. To distinguish between cameras of the same model imperfections of camera the use of hardware features seems to be the best method. Methods for forgery detection also rely on hardware-dependent characteristics but show a lower accuracy rates compared to camera identification methods. In another survey, Weiqi et al. \cite{luo2007survey} describe methods for passive technology for digital image forensics. They state that in most cases passive forensics can be converted to a problem of pattern recognition.

In forgery detection methods to identify JPEG quantization tables are often used. In reseach by Kornblum\cite{kornblum2008using} quantization tables used by several image software are identified. A software library called Calvin is developed to identify those images who cannot
be guaranteed to have been created by a real camera. Reseach by Farid\cite{4773149} shows a technique for detecting tampering in low-quality JPEG images by identifying a cumulative effect of quantization.

JPEG quantization tables can also be used for source identification. Farid has performed research\cite{farid1}\cite{farid2008digital} on source identification with the use of JPEG quantization tables. This research states that a sort of camera signature is embedded within each JPEG image due to the used JPEG quantization tables since they differ between manufacturers. Although the JPEG quantization is not perfectly unique, the majority of cases where the same tables are found it is cameras from the same manufacturer that share the same quantization table. It states that (the use of JPEG quantization tables) ``\textit{is reasonably effective at narrowing the source of an image to a single camera make and model or to a small set of possible cameras.}" (p. 3)

There exist several projects where JPEG quantization tables are used as camera signatures. For example, the JPEGsnoop\footnote{http://www.impulseadventure.com/photo/jpeg-snoop.html} project reports a huge amount of information to expose hidden information in images. Another project is the (discontinued) commercial FourMatch\footnote{http://www.fourandsix.com/fourmatch}, which was focused on forgery detection. These projects are not focused on matching large sets of images against a large camera database. In contrast, this research hopes to contribute by creating a decision tree model in order to decrease the search space for large datasets and which can easily be combined further with other (more accurate) source identification techniques. 

\section*{Approach \& Methods}

In order to optimize search through JPEG quantization tables the search space needs to be decreased. This reduction in search space can be performed by creating a decision tree model. This model maps observations about an item (specific features of the quantization table) to conclusions about the item's target value (camera model). Decision tree learning is used, which is the construction of a decision tree from class-labelled training tuples, to identify important parameters and their position in the decision tree model. The matching with the use of decision tree model parameters and the matching between full JPEG quantization tables are both benchmarked for time to see whether the search time is accelerated.
\pagebreak

\noindent The following steps are taken:
\begin{enumerate}
\item Gather dataset of JPEG quantization tables. Dataset of pictures and their JPEG quantization table and for JPEG quantization for camera models are needed.
\item Create numerous possible parameters to identify these tables. Rewrite JPEG quantization table as collection of these parameter values.
\item Create training and test set for decision tree learning.
\item Perform decision tree learning to create decision tree model
\item Perform benchmarks: matching with the decision tree model parameters and matching full JPEG quantization tables
\end{enumerate}


\subsection*{Requirements}

For this project the following requirements are made:

\begin{itemize}
\item Large database of photos taken from many different cameras
\item Database of JPEG quantization tables from many different cameras
\end{itemize}

\noindent These databases are needed to analyse JPEG quantization tables. They are also needed to create a training and test set to perform the decision tree learning algorithm.
\section*{Planning}
The planning for this research is proposed in Table \ref{table:planning}.

\begin{table}[H]
\small
\caption{Project planning including dates and tasks}
\label{table:planning}
%\resizebox{0.75\textwidth}{!}{\begin{minipage}{\textwidth}
\begin{tabular}{|l l|l l|}

\hline
\multicolumn{2}{|l|}{\textbf{Week 1 (2-8/06)}} & \multicolumn{2}{|l|}{\textbf{Week 2 (9-15/06)}}\\
\hline
& Literature research & & Gather databases \\
& Write project proposal & & Create parameters  \\

\textbf{6/06} & Deadline project proposal & & Start decision tree learning \\ 
\hline
\multicolumn{2}{|l|}{\textbf{Week 3 (16-22/06)}} & \multicolumn{2}{|l|}{\textbf{Week 4 (23-29/06)}}\\
\hline
& Continue decision tree learning & & Write report\\
& Benchmark methods & & Start presentation\\
\hline
\multicolumn{2}{|l|}{\textbf{Week 5 (30/06 - 06/07)}} & & \\
\hline
& Write report &  \multicolumn{2}{|c|}{~} \\
\textbf{2/07} & Presentation &   \multicolumn{2}{|c|}{~}\\
\textbf{4/07} & Deadline report &  \multicolumn{2}{|c|}{~} \\
\hline

\hline
\end{tabular}
%      \end{minipage}}

\end{table}

\vspace{-2em}
\section*{Expected Product}
The expected product of this research will be a decision tree model which can be used to quickly decrease the search space for identifying the origin of images to a specific camera with the use of JPEG quantization tables.

\section*{Ethical Considerations}
In this research no personally identifiable information is used. The photograph databases will not contain sensitive photographs and  are solely researched on their JPEG quantization table and not on their content. An example database is the \textit{'Dresden Image Database'}\footnote{http://forensics.inf.tu-dresden.de/ddimgdb}, which is specifically built for the purpose of development and benchmarking of camera-based digital forensic techniques.

\pagebreak

\bibliography{bibliography}{}
\bibliographystyle{plain}

\end{document}
